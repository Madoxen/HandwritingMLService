\documentclass{article}
\usepackage[utf8]{inputenc}
\usepackage{polski}
\usepackage{mathtools}
\usepackage[margin=2cm]{geometry}
\usepackage{listings}
\usepackage{color}
\definecolor{lightgray}{rgb}{.9,.9,.9}
\definecolor{darkgray}{rgb}{.4,.4,.4}
\definecolor{purple}{rgb}{0.65, 0.12, 0.82}


\title{Dokumentacja Projektu \\ Języki Skryptowe \\ Klasyfikator ręcznego pisma}
\author{Chłąd Paweł}
\date{2019 \\ Grudzień}



\lstdefinelanguage{JavaScript}{
  keywords={typeof, new, true, false, catch, function, return, null, catch, switch, var, if, in, while, do, else, case, break},
  keywordstyle=\color{blue}\bfseries,
  ndkeywords={class, export, boolean, throw, implements, import, this},
  ndkeywordstyle=\color{darkgray}\bfseries,
  identifierstyle=\color{black},
  sensitive=false,
  comment=[l]{//},
  morecomment=[s]{/*}{*/},
  commentstyle=\color{purple}\ttfamily,
  stringstyle=\color{red}\ttfamily,
  morestring=[b]',
  morestring=[b]"
}


\lstset{frame=tb,
  breaklines=true,
  showstringspaces=false,
  columns=flexible,
  numbers=none,
  tabsize=4
}


\begin{document}
\maketitle
\pagebreak
\section{Opis działania}

\subsection{Wstęp}
Główny program składa się z serwera HTTP, którego
zadaniem jest obsługa użytkownika. Drugą częścią jest 
strona internetowa która stanowi interfejs pomiędzy serwerem
a użytkownikiem. Trzecią częścią jest program uczący, za jego
pomocą została wytrenowana sieć neuralna, która potrafi rozpoznawać
napisane przez użytkownika liczby. Sprawność sieci wynosi 95\% ale 
ze względu na brak normalizacji danych wejściowych, procent ten będzie niższy
dla prawdziwych (nieznormalizowanych) danych. 

\subsection{Instrukcja Obsługi}

Program uruchamiamy przez Launch.bat albo server.exe. Launch.bat przedstawi nam dodatkowe opcje:
\begin{itemize}
    \item Start server - uruchomi server.exe
    \item Backup - utworzy kopię sieci neuralnej i zebranych zdjęć
    \item Exit - wyjdzie z pliku wsadowego
\end{itemize}




\section{Strona techniczna}
\subsection{Struktura}
Serwer HTTP oraz program uczący zostały zrealizowane w języku Python (ver. 3.7.5).
Strona internetowa w języku markupowym HTML oraz skrypty w języku JavaScript.

Wszelkie pliki źródłowe znajdują się pod adresem: https://github.com/Madoxen/HandwritingMLService

Lista użytych bibliotek:
\begin{itemize}
    \item numpy
    \item PIL (Python Imaging Library)
\end{itemize}



Program ma następującą strukturę
\begin{itemize}
    \item bin - folder zawierający pliki wykonywalne
    \item src - folder zawierający repozytorium git
    \item backup - folder zawierający kopię zapasową danych do nauki i repozytorium git
\end{itemize}

\subsection{Opis kodu}
\subsubsection{server.py}
Moduł python zawierający klasę Server. Klasa ta odpowiada za uruchamianie i wstrzymywanie faktycznego serwera HTTP (http.server) oraz jego handler'a (http\_handler).

Klasa Server zawiera następujące właściwości:
\begin{itemize}
    \item \textbf{addr} - tuple zawierający adres i port serwera
    \item \textbf{server} - instancja serwera http
    \item \textbf{server\_thread} - instancja wątku serwera
    \item \textbf{restarting} - flaga oznaczająca restart serwera
\end{itemize}

Klasa Server zawiera następujące metody:
\begin{itemize}
    \item \textbf{\_\_init\_\_(addr)} - konstruktor klasy Server; Addr - krotka (tuple) złożona z adresu IP oraz portu, na którym ma nasłuchiwać serwer.
    \item \textbf{run()} - metoda która uruchamia wątek serwera i oczekuje wprowadzenia potencjalnej komendy od użytkownika na wątku głównym
    \item \textbf{server\_loop()} - metoda która jest używana podczas konstrukcji wątku serwera 
\end{itemize}


\subsubsection{http\_handler.py}
Moduł python zawierający klasę webServerHandler, której bazą jest klasa http.server.BaseHTTPRequestHandler. webServerHandler odpowiada za obsługę zapytań GET i POST.

Klasa webServerHandler zawiera następujące właściwości:
\begin{itemize}
    \item \textbf{root} - ścieżka do folderu zawierającego serwowaną stronę internetową
\end{itemize}

Klasa webServerHandler zawiera następujące metody:
\begin{itemize}
    \item \textbf{do\_GET()} - metoda wywoływana podczas zapytania GET - serwuje stronę i jej skrypty znajdujące się w root
    \item \textbf{do\_POST()} - metoda wywoływana podczas zapytania POST - obsługuje wymianę danych dot. wpisanego numeru. Użytkownik klikając przycisk "wyślij" wysyła dane o stworzonym przez siebie
    obrazie, w pliku JSON. Następnie plik ten jest czytany przez serwer, który następnie dokonuje klasyfikacji przy użyciu już wytrenowanej sieci neuralnej.
\end{itemize}


\subsubsection{nnetwork.py}
Moduł python zawierajacy klasę Network. Klasa ta zawiera w sobie dane o sieci neuralnej oraz algorytm uczenia.

Sieć neuralna składa się z czterech warstw:
\begin{itemize}
    \item 784 neurony wejściowe (obraz 28x28 pix, wartości od 0 - 1 gdzie 1 to zarysowany obszar)
    \item Dwie warstwy ukryte po 30 neuronów
    \item 10 neuronów wyjściowych (oznaczające pewność klasyfikacji dla cyfr od 0 do 9)
\end{itemize}


Do nauczenia sieci został wykorzystany algorytm \textit{gradientu prostego (gradient descend)} połączony z algorytmem \textit{propagacji wstecznej (backpropagation)}.


\textbf{Gradient prosty} 
Metoda gradientu prostego opiera się na prostej obserwacji, nasza sieć neuralna to tak naprawdę pewna skomplikowana funkcja. Problem polega na tym, że 
nie możemy zminimalizować sieci bezpośrednio (co by to w ogóle znaczyło?), natomiast możemy zminimalizować pewną inną powiązaną funkcję. Taką funkcję
nazywa się funkcją kosztu (eng. cost/loss function). Funkcją kosztu może być dowolna funkcja, która będzie wskazywała na pewien "odchył" od prawidłowego stanu.
\newline
W przypadku programu funkcją kosztu jest funkcja kosztu kwadratowego:
\begin{eqnarray}  
    C(w,b) \equiv \frac{1}{2n} \sum_x \| y(x) - a\|^2.
\end{eqnarray}

Gdzie:
\begin{itemize}
    \item $w$ - wagi
    \item $b$ - biasy
    \item $y(x)$ - prawidłowe wartości na neuronach wyjściowych (wektor 10 wymiarowy)
    \item $a$ - rzeczywiste wartości powstałe na neuronach wyjściowych (wektor 10 wymiarowy) 
\end{itemize}
Musimy więc znaleźć minimum funkcji kosztu, tradycyjna metoda niestety tutaj nie zadziała, gdyż funkcja ta w przypadku tego programu będzie miała tysiące zmiennych,
dlatego też użyjemy metody przybliżonej. Wyliczając gradient takiej funkcji możemy łatwo określić \textit{kierunek} w którym musimy się poruszać, aby dotrzeć do jej minimum.
\begin{eqnarray} 
    \nabla C \equiv \left( \frac{\partial C}{\partial v_1}, 
    \frac{\partial C}{\partial v_2} ... \frac{\partial C}{\partial v_n} \right)^T.
\end{eqnarray}

\pagebreak
Pozwala nam to na określenie wektora zmian

\begin{eqnarray} 
    \Delta v = -\eta \nabla C,
\end{eqnarray}

gdzie $\eta$ to parametr uczenia (tzn. jak daleko przemieścimy się w następnym kroku). To wszystko przekłada się na generalną "zasadę" poruszania się w kierunku minimum.
\begin{eqnarray}
    v \rightarrow v' = v-\eta \nabla C.
\end{eqnarray}

Więc w naszym przypadku przekładamy to na odpowiednie wagi i biasy 
\begin{eqnarray}
    w_k & \rightarrow & w_k' = w_k-\eta \frac{\partial C}{\partial w_k} \\
    b_l & \rightarrow & b_l' = b_l-\eta \frac{\partial C}{\partial b_l}
\end{eqnarray}


Klasa Network zawiera następujące właściwości:
\begin{itemize}
    \item \textbf{num\_layers} - liczba warstw neuronów
    \item \textbf{sizes} - tuple zawierający liczbę neuronów w każdej warstwie
    \item \textbf{biases} - lista, zawierająca tablice (numpy) biasów odpowiednich warstw (pomijając warstwę wejściową)
    \item \textbf{weights} - lista, zawierająca tablice (numpy) wag odpowiednich warstw, np. dla warstwy (10) i warstwy (2) tablica ta będzie miała wymiar 10x2 gdyż do każdego neuronu
    musimy powiązać każdy neuron z następnej warstwy
\end{itemize}

Klasa Network zawiera następujące metody:
\begin{itemize}
    \item \textbf{\_\_init\_\_(sizes=None, path=None)} - konstruktor klasy Network; dołączenie \textit{sizes} spowoduje utworzenie tablic o zadanych rozmiarach, z losowymi elementami (losowymi w rozkładzie gaussa pomiędzy 0 a 1);
    dołączenie argumentu path spowoduje próbę odczytania sieci z zadanego pliku. (Formatem sieci jest prosty plik JSON)
    \item \textbf{feedforward(a)} - oblicza wektor wyjściowy na podstawie zadanego wektora wejściowego
    \item \textbf{SGD(training\_data, epochs, mini\_batch\_size, eta, test\_data=None)} - wykonuje algorytm prostego gradientu; training\_data - lista tupli, w których przechowywane są parami wektory wejściowe i poprawne wektory wyjściowe;
    epochs - ilość powtórzeń; mini\_batch\_size - rozmiar małych paczek z danymi wejściowymi (ilość w tuplach na paczkę), ustawienie tego parametru na 1 spowoduje wykorzystanie zwykłego
    algorytmu gradientu prostego, ustawienie parametru na wielkość większą od jeden spowoduje użycie przybliżonego algorytmu gradientu prostego; eta - szybkość uczenia; test\_data - tak samo jak training\_data, jeśli zostanie podane
    spowoduje to, że w każdym epochu/powtórzeniu sieć zostanie przetestowana, a postęp nauki wyświetlany co powtórzenie.
    \item \textbf{backprop(x,y)} - funkcja obliczająca $\frac{\partial C}{\partial w_k}$ oraz $\frac{\partial C}{\partial b_l}$ stosując algorytm propagacji wstecznej; x - wektor wejściowy; y - spodziewany wektor wyjścia.
    \item \textbf{evaluate(test\_data)} - funkcja sprawdzająca liczbę poprawnych ewaluacji sieci neuralnej.
    \item \textbf{cost\_derivative(output\_activations, y)} - oblicza wartość pochodnej dla funkcji kosztu.
\end{itemize}


\subsubsection{ml\_service.py}
Moduł python zawierający klasę MLService. Klasa ta jest odpowiedzialna za przygotowywanie informacji wejściowych.

Klasa MLService zawiera w sobie jedną właściwość statyczną:
\begin{itemize}
    \item \textbf{net} - instancja klasy Network
\end{itemize}

Klasa MLService zawiera w sobie następujące metody:
\begin{itemize}
    \item \textbf{prepareImage(img\_data)} - przygotowuje obraz w gotowy do użycia wektor 784 wymiarowy; img\_data - dane obrazu w formie JSON
    \item \textbf{evaluate(img\_data)} - zwraca wektor wyjściowy; img\_data - wektor 784 wymiarowy z danymi obrazu.
\end{itemize}


\pagebreak
\lstinputlisting[language=Python]{../api/server.py}
\pagebreak
\lstinputlisting[language=Python]{../api/ml_service.py}
\pagebreak
\lstinputlisting[language=Python]{../api/nnetwork.py}
\pagebreak
\lstinputlisting[language=Python]{../api/http_handler.py}
\pagebreak
\lstinputlisting[language=HTML]{../site/index.html}
\pagebreak
\lstinputlisting[language=JavaScript]{../site/scripts/drawer.js}
\pagebreak
\lstinputlisting[language=JavaScript]{../site/scripts/main.js}
\pagebreak
\lstinputlisting[language=JavaScript]{../site/scripts/point.js}
\pagebreak
\lstinputlisting{../api/launch.bat}


\end{document}

